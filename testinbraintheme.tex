\documentclass{beamer}

\title[]{InBrain Beamer Template -- v 1.0}

\author{Bruno Hebling Vieira}
\institute[FFCLRP-USP]{
InBrain Lab, Faculdade de Filosofia, Ciências e Letras de Ribeirão Preto\\Universidade de São Paulo \\
\medskip
\textit{bruno.hebling.vieira@usp.br} \\
\medskip
\texttt{inbrainlab.com}
}
\date{\today}


\usetheme{inbrain}


\usepackage[style=authoryear,backend=bibtex8,doi=false,isbn=false,url=false,eprint=false]{biblatex}
\renewbibmacro{in:}{}
\AtEveryCitekey{%
    \clearfield{volume}%
    \clearfield{number}%
    \clearfield{title}%
    \clearfield{pages}}
\addbibresource{library.bib}


\begin{document}

\begin{withoutheadline}
    \begin{frame}
    \titlepage % Print the title page as the first slide
    \end{frame}
\end{withoutheadline}

%------------------------------------------------
\section[Intro]{Introduction}
%------------------------------------------------

\begin{frame}
    \frametitle{Intelligence}

    ``[\dots] the aggregate or global capacity of the individual to act purposefully, to think rationally and to deal effectively with his environment''\footnotemark[1]

    
    \begin{itemize}
        \item It comprises several distinct aptitudes
        \item Tests were developed to quantify intelligence, making it is possible to assess a score in a numerical cognitive scale
    \end{itemize}

    \footnotetext[1]{\fullcite{Wechsler1944}}
\end{frame}

\begin{frame}
    \frametitle{General Intelligence}
    
    \begin{itemize}
        \item Since test scores tend to correlate positively (i.e. a positive manifold), this led to the development of theories on general intelligence, or ``g''\footnotemark[1]
        \item ``G'' is an underlying factor that influences the overall cognitive performance of an individual and is latent to intelligence tests
        \item It can be estimated from the factor analytic decomposition of  test scores from a population\footnotemark[2]
    \end{itemize}

    \footnotetext[1]{\fullcite{Spearman1904}}
    \footnotetext[2]{\fullcite{Thurstone1940}}
\end{frame}


%------------------------------------------------
\section[Methods]{Materials \& Methods}
%------------------------------------------------

\begin{frame}
    \frametitle{Materials \& Methods}

    \begin{columns}[c]
        \begin{column}{.4\textwidth}
            \begin{itemize}
                \item A
                \item B
                \item C
            \end{itemize}
        \end{column}
        \begin{column}{.4\textwidth}
            \begin{enumerate}
                \item 1
                \item 2
                \item 3
                \item 4
            \end{enumerate}
        \end{column}
    \end{columns}

\end{frame}


%------------------------------------------------
\section[Conclusions]{Conclusions}
%------------------------------------------------

\begin{frame}
    \frametitle{Final remarks}

    Carpe diem.

\end{frame}

%------------------------------------------------
% Final slides
%------------------------------------------------

\begin{frame}
    \centering
    {\huge``É o InBrain ou não é?!''}
    
    J.H.O. Barbosa \textit{``InBrain Archives''}. 2012
\end{frame}

\begin{frame}
    \centering
    \Huge{Thank you!}
\end{frame}

\end{document}
